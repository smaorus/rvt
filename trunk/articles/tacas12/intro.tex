\section{Introduction} \label{sec:intro}

Whereas termination of a single program has been widely studied~(e.g.,
\cite{Fl67,CPR05,BMS05,CPR11}) for several decades by now, with the focus
being, especially in the last few years, on automating such proofs,
little attention has been paid to the related problem of proving that
two similar programs (e.g., two consecutive versions of the same
program) terminate on exactly the same inputs.  Ideally one should
focus on the former problem, but this is not always possible either
because the automatic techniques are inherently incomplete, or because
\emph{by design} the program does not terminate on all inputs. In such
cases there is value in solving the latter problem, because
developers may wish to know that none of their changes affect the
termination behavior of their program. Moreover, the problem and
solution thereof can be defined in the granularity of functions rather
than whole programs; in this case the developer may benefit even more from
a detailed list of pairs of functions that terminate on exactly the
same set of inputs. Those pairs that are not on the list can help
detecting termination errors.

Our focus is on successive, closely related versions of a program because it
both reflects a realistic problem of developers, and offers opportunities for
decomposition and abstraction that are not possible with the single-program
termination problem.  This problem, which was initially proposed in~\cite{GS08}
and coined \emph{mutual termination}, can easily be proven undecidable as can
be seen via a simple reduction from the halting problem. We argue, however,
that in many cases it is easier to solve automatically, because unlike termination proofs for a single program, it does not rely on proving that
the sequence of states in the programs' computations can be mapped into a well-founded
set. Rather it can be proven by showing that the loops and recursive functions
have the same set of function calls given the same inputs, which is relatively
easier to prove automatically. In Sec.~\ref{sec:rule}, for example, we show how to prove mutual termination of two versions of the famous Collatz's $3x + 1$ problem~\cite{G81}; whereas proving termination of this program is open for many decades, proving mutual termination with respect to another version is simple.

Our suggested method for decomposing the proof is most valuable when the two
input programs $P$ and $P'$ are relatively similar in structure. In fact, its
complexity is dominated by the \emph{difference} between the programs, rather
than by their absolute size. It begins by heuristically building a (possibly
partial) map between the functions of $P$ and $P'$. It then progresses
bottom-up on the two call graphs, and each time proves the mutual termination
of a pair of functions in the map, while abstracting their callees. The
generated verification conditions are in the form of assertions about `flat' programs (i.e., without
loops and recursive calls), which are proportional in size to the two compared
functions. It then discharges these verification conditions with a bounded
model-checker (CBMC~\cite{CK03} in our case). Each such program has the same
structure: it calls the two compared functions sequentially with the same
nondeterministic input, records all subsequent function calls and their
arguments, and asserts in the end that they have an equivalent set of function
calls. According to our proof rule, the validity of this assertion is
sufficient for establishing their mutual termination.


The algorithm is rather involved because it has to deal with cases in which the
call graphs of $P$ and $P'$ are not isomorphic (this leads to unmapped
functions), with mutually recursive functions, and with cases in which the
proof of mutual termination for the callees has failed. It also improves completeness by utilizing extra
knowledge that we may give to it on the \emph{partial equivalence} of the
callees, where two functions are said to be partially equivalent if given the
same inputs they terminate with the same outputs, or at least one of them does
not terminate.
%If we know that two mapped callees are partially equivalent, we
%abstract them with the same uninterpreted function,
%which increases our chance
%to prove mutual termination.
Partial equivalence was studied in~\cite{GS08,GS09} and is implemented in \tool{rvt}~\cite{GS09} and Microsoft's
\tool{SymDiff}~\cite{KLR10}. We also implemented our algorithm in \tool{rvt}, which enables us to gain this information in a preprocessing step.


To summarize our contributions in this paper, we present a) a proof rule for inferring mutual termination of recursive (and mutually-recursive) functions at the leaves of their respective call graphs,
%, which requires a weaker premise than the one in~\cite{GS10}
 b) an extension of the first rule that applies also to internal nodes in the call graphs, and c) a proof rule for inferring \emph{termination} (not mutual termination) in case the other function is known to be terminating. More importantly, we show how these rules can be applied to whole programs via a bottom-up decomposition algorithm, and report on a prototype implementation of this algorithm -- the first to deal with the mutual termination problem.

%In the next section we give a formal definition of our problem and a description of the preprocessing that we %apply to programs in order to be able to use our proof rules and decomposition algorithm.
%Section~\ref{sec:rule} describe a proof rule for proving mutual termination of functions in mutual recursion. Section~\ref{sec:decomposition} suggests a method for applying the rule for whole programs, based on a bottom-up traversal of the two call graphs. In Sec.~\ref{sec:term} we consider a related problem: \emph{assuming} that $P$ terminates, prove that $P'$ terminates. Experiments and conclusions are summarized in Sec.~\ref{sec:experiments}.
