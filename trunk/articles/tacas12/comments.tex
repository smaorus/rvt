
\begin{example} \label{ex:mterm}
Consider the following two programs, which are clearly not partially equivalent:

\vspace{0.2 cm}
\begin{minipage}{5.5 cm}
\begin{algorithmic}
\Function{$f$}{unsigned $a$}

\If {$a == 0$} \State \Return 1;

\EndIf

\Return $a * f(a - 1)$;

\EndFunction
\end{algorithmic}
\end{minipage}
\begin{minipage}{5.5 cm}
\begin{algorithmic}
\Function{$f'$}{unsigned $a'$}

\If {$a' == 0$} \State \Return 2;

\EndIf

\State \Return $a' * f'(a' - 1)$;

\EndFunction
\end{algorithmic}
\end{minipage}
\vspace{0.2 cm}
%

\noindent
We can prove their mutual termination nevertheless, by proving the call equivalence of \upbodiesand{f}:

\vspace{0.2 cm}
\begin{minipage}{5.5 cm}
\begin{algorithmic}
\Function{$\upbody{f}$}{unsigned $a$}

\If {$a == 0$} \State \Return 1 \EndIf

\State \Return $a * \uf{f}( a - 1 )$;

\EndFunction
\end{algorithmic}
\end{minipage}
%
\begin{minipage}{5.5 cm}
\begin{algorithmic}
\Function{$\upbody{f'}$}{unsigned $a'$}

\If { $a' == 0$ } \State \Return 2; \EndIf

\State \Return $a' * \uf{f'}( a' - 1 )$;

\EndFunction
\end{algorithmic}
\end{minipage}
\vspace{0.2 cm}
%

\noindent Note that now $\uf{f}$ and $\uf{f'}$ are different. According to
(\ref{eq:new}), we need to prove, using once more the proof-theoretic approach,
the validity of:

\[ \begin{array}{l}
(T_{\upbody{f}}\ \land\ T_{\upbody{f'}}\ \land\ a = a')\ \rightarrow \\
\big((a \neq 0 \leftrightarrow a' \neq 0)\ \land\ (a \neq 0\ \rightarrow\ a-1 = a'-1)\big)\;,
\end{array}\]
%
which is, indeed, a valid formula. In this case, we are able to prove
termination without partial equivalence because the return values
of $\uf{f}$ and $\uf{f'}$ affect neither the guarding conditions nor the input arguments of other function calls. \qed
\end{example}


%%%%%%%%%%%%%%%%%%%%%%%%%%%%%%%%%%%%%%%% ATVA

\comment{
\section{A weaker premise}\label{sec:enhancement}
The mutual termination rule \mtermg requires $\pe(f,f')$ for each
$\pair{f,f'}\in \mapf(m)$. We now argue that $\requiv(f,f')$ is sufficient.
Partial equivalence still plays a role, but only in improving completeness.
Specifically, for callees $\pair{g,g'} \in \mapf(m)$ of $\pair{f,f'}$,
respectively, having $\pe(g,g')$ allows us to replace them with the same
uninterpreted functions. Otherwise we need to replace them with different
uninterpreted functions, which is still sound for proving mutual termination,
but hinders completeness. This change is reflected in the way we
construct $\upbody{f}$ and $\upbody{f'}$. Specifically, the enforcement stated
in (\ref{eq:enforcement}) is now weaker:
\begin{equation}\label{eq:enforcementnew}
\left(\pair{g, g'} \in\mapf \wedge \pe(g, g')\right) \Leftarrow \uf{g} = \uf{g'} \mbox{\ \ \ \ \enforcementnew}\;.
\end{equation}
%
In words, $\uf{g} = \uf{g'}$ only if  $\pe$(\comtag{g}) is proven. Rather than having to prove it for each pair in $\mapf(m)$, then, we construct $\upbody{f},\upbody{f'}$ based on a set of partially-equivalent pairs that are given to us as input. The more pairs from $m$ this set contains, the better chance we have to prove mutual termination.

Using the new definition of $\upbody{f},\upbody{f'}$, the new rule is:
\begin{equation} \label{eq:new}
\frac
{\forall \pair{f, f'} \in\mapf(m).\ \requiv(\upbody{f},\upbody{f'})}
{\forall \pair{f, f'} \in\mapf(m).\ \mt(f, f')}
\mbox{\ \mtermd} \;.
\end{equation}
%
The proposed rule \mtermd has a weaker premise than \mtermg and can therefore
prove mutual termination in more cases.
The following example demonstrates how \mtermd can be applied.
}





\comment{
We now show an example in which the prior knowledge on the partial equivalence of the callees is necessary for establishing call equivalence.

\begin{example} \label{ex:mccarthy}
Consider the two implementations of the McCarthy\mbox{-91}
function~\cite{MP70} in Fig.~\ref{fig:mccarthy} (top).
%
We naturally define the mapping $\mapf = \{\pair{f,f'}\}$. It is not
hard to see that $f,f'$ are partially equivalent, and, indeed, assume that it
has been proven. Next, we
need to prove call equivalence of $\upbody{f}$ and $\upbody{f'}$. Since we
enforce $\uf{f} = \uf{f'}$ (see (\ref{eq:enforcement})), in the listing at the bottom part of the figure they appear as the same uninterpreted function.


\begin{figure}
\begin{tabular}{p{6 cm}|p{6 cm}} \hline
\begin{algorithmic}
\Function{$f$}{int $a$}

\If {$a > 100$} \State \Return $a - 10$; \EndIf

\State \Return $f(f( a + 11 ))$;

\EndFunction

\end{algorithmic}
&
\begin{algorithmic}
\Function{$f'$}{int $a'$}

\If {$a' \leq 100$ } \State \Return $f'(f'( a' + 11 ))$; \EndIf

\State \Return $a' - 10$;

\EndFunction
\end{algorithmic} \\[-6pt] \hline

\begin{algorithmic}
\Function{$\upbody{f}$}{int $a$}

\If {$a > 100$ }

\State \Return $a - 10$; \EndIf

\State \Return $\uf{f}(\uf{f}( a + 11 ))$;

\EndFunction
\end{algorithmic}
&
\begin{algorithmic}
\Function{$\upbody{f'}$}{int $a'$}

\If {$a' \leq 100$ }

\State \Return $\uf{f}(\uf{f}( a' + 11 ))$; \EndIf

\State \Return $a' - 10$;

\EndFunction
\end{algorithmic} \\ \hline
\end{tabular}
\caption{Two implementations of the McCarthy-91 function (top), and their isolated versions (bottom).}\label{fig:mccarthy}

\end{figure}

A \emph{proof-theoretic} method for establishing $\requiv(\upbody{f},
\upbody{f'})$ amounts to deriving from this code a formula -- a verification
condition -- which is valid iff these functions are call-equivalent. In this
case such a formula is:
\begin{equation} \label{eq:vc}
\begin{array}{l}
(T_{\upbody{f}} \ \land\ T_{\upbody{f'}}\ \land\ a = a')\ \rightarrow \\
\big((\lnot (a > 100)\ \leftrightarrow\ (a' \leq 100))\ \land \hspace{3.5 cm} \triangleright \mbox{ equal guards}\\
(\lnot (a > 100))\ \rightarrow\ \hspace{5.73 cm} \triangleright \mbox{ if called, then  }  \\
(a + 11 = a' + 11 \land \uf{f}(a+11) = \uf{f}(a' + 11))\big)\;, \hspace{0.50 cm} \triangleright \mbox{ equal arguments}
\end{array}
\end{equation}
%
where $T_{\upbody{f}},T_{\upbody{f'}}$ denote respectively the transition
relation of $\upbody{f}$ and $\upbody{f'}$. The last
line in (\ref{eq:vc}) corresponds, respectively, to the arguments of the inner
and outer calls to $f,f'$. This formula is valid, and its validity relies on
the fact that $\uf{f} = \uf{f'}$. This demonstrates that establishing
$\pe(f,f')$ can be essential.

The McCarthy-91 function is frequently used in the program-termination literature to demonstrate how a ranking function can be found. As we have shown here, for mutual termination no such ranking function is necessary.
\qed
\end{example}

}


\comment{
The proof-theoretic method demonstrated in the above example has the
disadvantage that it requires program analysis in order to derive the
conditions. We suggest instead a \emph{model-theoretic} method, which delegates
most of the analysis to an off-the-shelf model-checker. Instead of analyzing
the code to derive a general formula expressing the conditions of calls and the
actual arguments, derived programs are produced that would record all arguments
to the relevant functions. These programs are never actually executed. Instead,
an assertion of equivalence between the sets of arguments with which the functions are
called is model-checked, showing it true in every possible computation, and
thus automatically detecting call-equivalence. We will elaborate on this method
in Sect.~\ref{sec:decomposition}.
} 



%%%%%%%%%%%%%%%%%%%%%%%%%%%%%%%%%%%%%%%%% old
\comment{
\begin{figure}
\begin{tabular}{p{6 cm}p{6 cm}} \hline
\begin{algorithmic}
\Function {$f^{UF}$}{int $a$}

\If{$a > 1$}

\If {$\lnot a \% 2$} %{$\lnot$($a \& 1$)}

\State $a := a/2$;

\Else\ $a := 3a + 1$;

\EndIf

\State \alg{UF}$(f,a)$;
\EndIf
\State \Return 0;

\EndFunction
\end{algorithmic}
&
\begin{algorithmic}
\Function {$f'^{UF}$}{int $a'$}

\If{$a' \leq 1$} \Return 1; \EndIf

\State $t' := a' / 2$;

\If {$a'\% 2$}

\State $a' := 6t' + 4$;

\Else\ $a' := t'$;

\EndIf

\State \alg{UF}$(f',a')$;

\State \Return 1;
\EndFunction
\end{algorithmic} \\ %\hline
}
%%%%%%%%%%%%%%%%%%%% end old

%%%%%%%%%%%%%%%%%%%%%%%%%%%%%%%%%%%%%%%%%%%%%%%%%%%%%%%%%%%  old
\comment{
\begin{figure}
\begin{tabular}{p{6 cm}p{6 cm}} \hline
\begin{algorithmic}
\Function {$f$}{int $a$}
\If{$a > 1$}
    \If {$\lnot (a \% 2)$} \Comment{even } %{$\lnot$($a \& 1$)}
        \State $a := a/2$;
        \Else\ $a := 3a + 1$;
    \EndIf
    \State \alg{$f$}($a$);
\EndIf
\Return 0;

\EndFunction
\end{algorithmic}
&
\begin{algorithmic}
\Function {$f'$}{int $a'$}

\If{$a' \leq 1$} \Return 1\EndIf

\State $t' := a' / 2$; %\Comment{Divide by 2}

\If {$a'\% 2$} \Comment{odd}

\State $a' := 6t' + 4$;

\Else\ $a' := t'$;

\EndIf

\State \alg{$f$}($a'$);

\State \Return 1;

\EndFunction
\end{algorithmic} \\[-6pt] \hline
\end{tabular}
\caption{Two different implementations of the Collatz (``$3x + 1$") function that are mutually terminating.}\label{fig:collatz}
\end{figure}
}


%%%%%%%%%%%%%%%%%%%%%%%% end old


\comment{DDD
\Function{isolate}{functions \comtag{f}, function pairs $S$} \Comment{Builds
$\mtbody{f}, \mtbody{f'}$}
  \For {each $\{\pair{g,g'}\in \mapf \mid g,g'$ are reachable from $f,f'$\}}\label{step:forcheckCallequiv}
    \If {$\pair{g,g'} \in S$ or $\pair{g,g'}$ is marked \mtlabel} \label{step:substitute}
    \State \label{step:subst_uf}Replace calls to $g(\vec{in})$, $g'(\vec{in'})$, resp., with calls to:
     \begin{list} {$\bullet$}{\setlength{\leftmargin}{1.7cm}}
     \item \peuf{g}, \peufp{g'}, if $\pair{g,g'}$ is marked \equivlabel;
     \item \pneuf{g}, \pneufp{g'}, otherwise;
    \end{list}
    \Else { inline $g,g'$ in their callers; } \label{step:isinline}
    \EndIf
  \EndFor
  \State \Return $\pair{f,f'}$;
\EndFunction
\\}

\comment{DDD
\Function {CallEquiv}{A pair of isolated functions $\pair{\mtbody{f}, \mtbody{f'}}$}
\label{step:checkCallEquiv}
  \State \label{step:gen_main} Let $\delta$ denote the program:
  \[
  \begin{array}{l}
 \quad \quad \vec{in} := nondet(); \mtbody{f}(\vec{in}); \mtbody{f'}(\vec{in}); \\
 \quad \quad \mbox{{\sc uf()}; {\sc uf'()}; {\sc peuf()}; {\sc peuf'()}}; \hspace{3.7 cm} \triangleright \mbox { Defined in Fig.~\ref{fig:ufs},~\ref{fig:ufs1}}\\
 \end{array}
 \]
  \State \Return CBMC($\delta$); \label{step:cbmc}
\EndFunction

\\}